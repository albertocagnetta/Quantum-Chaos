%%%%% some useful environment

%comandi per enunciati


%\theoremstyle{plain}
\theorembodyfont{\normalfont}
\theoremheaderfont{\bfseries}
\theorempreskipamount\topsep
\theorempostskipamount\topsep
\theoremsymbol{$\clubsuit$}
%stile roman
\newtheorem{ex}{Esercizio}[section]
\newtheorem{ese}[ex]{Esempio} 
\newtheorem{eseex}[ex]{Es./Ex.} %definizione ambiente esempio/esercizio

\theoremsymbol{}

\theorembodyfont{\slshape}
\newtheorem{teo}{Teorema}[section] %definizione ambiente teorema
\newtheorem{teodef}[teo]{Thm./Def.} %definizione ambiente teorema
\newtheorem{prop}[teo]{Proposizione}    %definizione ambiente proposizione
\newtheorem{cor}[teo]{Corollario}       %definizione ambiente corollario
\newtheorem{defin}[teo]{Definizione}%definizione ambiente definizione 
\newtheorem{lem}[teo]{Lemma}           %definizione ambiente lemma 
\newtheorem{prob}[teo]{Problema}           %definizione ambiente lemma 

%\newenvironment{prf}{\begin{proof}{\textsl{Dimostrazione}}}{\end{proof}} % dimostrazioni con stile cambiato

% svolgimento
\theoremstyle{nonumberplain}
\theoremheaderfont{\slshape}
\theorembodyfont{\normalfont}
\theoremseparator{.\,}
\theoremsymbol{$\circledS$}
\newtheorem{svolg}{Svolgimento}
%

% dimostrazione
\theoremstyle{nonumberplain}
\theoremheaderfont{\slshape}
\theorembodyfont{\normalfont}
\theoremseparator{.\,}
\theoremsymbol{$\blacksquare$}
\newtheorem{prf}{Dimostrazione}
%


%Osservazione
\theoremstyle{plain}
\theoremheaderfont{\bfseries}
\theorembodyfont{\normalfont}
%\theoremseparator{.\,}
\theoremsymbol{}
\newtheorem{oss}[teo]{Osservazione}
%
%%%%%%%%



%comando teoremi importanti impTeo


%% the following is commaon for all examples in mdframed manual
%\mdfsetup{skipabove=\topskip,skipbelow=\topskip}
%%% up to here
\colorlet{ColimpTeo}{myroyal1}


\newenvironment{impTeo}[1][]{%
\refstepcounter{teo}%
\ifstrempty{#1}%
{\mdfsetup{%
frametitle={%
\tikz[baseline=(current bounding box.east),outer sep=0pt]
\node[anchor=east,rectangle,draw=ColimpTeo, fill=white, line width=1mm]
{\strut Teorema~\theteo};}}
}%
{\mdfsetup{%
frametitle={%
\tikz[baseline=(current bounding box.east),outer sep=0pt]
\node[anchor=east,rectangle,draw=ColimpTeo, fill=white, line width=1mm]
{\strut Teorema~\theteo:~#1};}}%
}%
\mdfsetup{innertopmargin=10pt,linecolor=ColimpTeo,%
linewidth=2pt,topline=true,%
frametitleaboveskip=\dimexpr-\ht\strutbox\relax
}
\begin{mdframed}[]\relax%
}{\end{mdframed}}






%\newenvironment{impTeo}[1][]{%
%\refstepcounter{teo}%
%%\setcounter{impTeo}{teo}%
%\ifstrempty{#1}%
% {\mdfsetup{%
%   frametitle={%
%    \tikz[baseline=(current bounding box.east),outer sep=0pt]
%    \node[anchor=east,rectangle,fill=blue!40]
%         {\strut Teorema~\theteo};}}
% }%
%{\mdfsetup{%
%  frametitle={%
%   \tikz[baseline=(current bounding box.east),outer sep=0pt]
%   \node[anchor=east,rectangle,fill=blue!40]
%        {\strut Teorema~\theteo:~#1};}}%
% }%
%\mdfsetup{innertopmargin=10pt,
%linecolor=blue!40,%
%       linewidth=2pt
%       ,topline=true
%       ,frametitleaboveskip=\dimexpr-\ht\strutbox\relax
%       }
%   \begin{mdframed}[]\relax%
%}
%{\end{mdframed}}




%% important Lemma 

\colorlet{ColimpLem}{myroyal1}

\mdfsetup{skipabove=\topskip,skipbelow=\topskip}
%%% upto here
\newenvironment{impLem}[1][]{%
\refstepcounter{teo}%
\ifstrempty{#1}%
 {\mdfsetup{%
   frametitle={%
    \tikz[baseline=(current bounding box.east),outer sep=0pt]
    \node[anchor=east,rectangle,draw=ColimpLem, fill=white, line width=1mm]
         {\strut Lemma~\theteo};}}
 }%
{\mdfsetup{%
  frametitle={%
   \tikz[baseline=(current bounding box.east),outer sep=0pt]
   \node[anchor=east,rectangle,draw=ColimpLem, fill=white, line width=1mm]
        {\strut Lemma~\theteo:~#1};}}%
 }%
\mdfsetup{innertopmargin=10pt,linecolor=ColimpLem,%
       linewidth=2pt,topline=true,
       frametitleaboveskip=\dimexpr-\ht\strutbox\relax,}
   \begin{mdframed}[]\relax%
}
{\end{mdframed}}






%%%%%% some useful commands

%comando virgolette
\newcommand{\virg}[1]{
\textquotedblleft #1\textquotedblright
}
%\renewcommand{\bm}[1]{\boldsymbol{#1}}



%\newcommand{\HRule}{\rule{\linewidth}{0.35mm}}

% Operatori matematici
\DeclareMathOperator{\Aut}{Aut}
\DeclareMathOperator{\Stab}{Stab}
\DeclareMathOperator{\Id}{Id}  
\DeclareMathOperator{\In}{\mathcal{U}}
\DeclareMathOperator{\tr}{tr}
\DeclareMathOperator{\tg}{tg}
\DeclareMathOperator{\ctg}{ctg}
\DeclareMathOperator{\arctg}{arctg}
\DeclareMathOperator{\arcctg}{arcctg}
\DeclareMathOperator{\sgn}{sgn}
\DeclareMathOperator{\Char}{char}
\DeclareMathOperator{\Gal}{Gal}
\DeclareMathOperator{\mcd}{mcd}
\DeclareMathOperator{\mcm}{mcm}
\DeclareMathOperator{\grad}{grad}
\DeclareMathOperator{\rot}{rot}
\DeclareMathOperator{\diag}{diag}
\DeclareMathOperator{\dom}{dom}
\DeclareMathOperator{\supp}{supp}
\DeclareMathOperator{\one}{\mathds{1}} %funzione caratteristica o vettore di tutti 1
\newcommand{\Hex}{\nabla^{2}}
%\newcommand{\cupdot}{\mathbin{\mathaccent\cdot\cup}}
%\newcommand{\bigcupdot}{\mathbin{\mathaccent\cdot\bigcup}}
\newcommand{\cupdot}{\dot{\cup}}
\newcommand{\bigcupdot}{\dot{\bigcup}}
\newcommand{\sumdot}{\dot{\sum}}
\newcommand{\N}{\mathbb{N}}
\newcommand{\Z}{\mathbb{Z}}
\newcommand{\Q}{\mathbb{Q}}
%\renewcommand{\C}{\mathbb{C}}
\newcommand{\R}{\mathbb{R}}
\newcommand{\X}{\chi}
%\newcommand{\=}{\stackrel{\text{def}}{=}}



% nuove variabili
\renewcommand{\i}{\mathbf{i}}
\renewcommand{\j}{\mathbf{j}}
\renewcommand{\k}{\mathbf{k}}
\newcommand{\D}{\,\text{d}} %differenziale
\newcommand{\vphi}{\varphi}
\newcommand{\veps}{\varepsilon}
\newcommand{\eps}{\epsilon}
\newcommand{\vtheta}{\vartheta}




%\DeclarePairedDelimiter\ceil{\lceil}{\rceil}
%\DeclarePairedDelimiter\floor{\lfloor}{\rfloor}
%\DeclarePairedDelimiter\fractional{\lbrace}{\rbrace}
\newcommand{\ceil}[1]{\left\lceil #1\right\rceil}
\newcommand{\floor}[1]{\left\lfloor #1\right\rfloor}
\newcommand{\fractional}[1]{\left\lbrace #1\right\rbrace}
%\newcommand{\dif}{\,\text{d}} % differenziale
%\renewcommand{\d}{\,\text{d}} % differenziale
\newcommand{\dff}{\coloneqq}
\newcommand{\df}{\displaystyle\frac} 
\newcommand{\measure}{\mathfrak{m}}
\newcommand{\abs}[1]{\left| #1\right|}
\newcommand{\norm}[1]{\left\lVert #1\right\rVert}
\newcommand{\pscl}[2]{\left< #1,#2\right>} 
\newcommand{\cpscl}[2]{\left(#1\,\middle|\,#2\right)} % prod scalare complesso
\newcommand{\spn}[1]{\left< #1\right>} %spanned
\newcommand{\vect}[1]{
\begin{pmatrix}
#1
\end{pmatrix}
}
\newcommand{\vv}[1]{\bm{#1}} %%comando stile vettori%
%%%%%