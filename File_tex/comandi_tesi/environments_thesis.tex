%%%%% some useful environment

%comandi per enunciati


%\theoremstyle{plain}
\theorembodyfont{\normalfont}
\theoremheaderfont{\bfseries}
\theorempreskipamount\topsep
\theorempostskipamount\topsep
\theoremsymbol{$\clubsuit$}
%stile roman
\newtheorem{ex}{Exercise}[section]
\newtheorem{ese}[ex]{Example} 
\newtheorem{eseex}[ex]{Es./Ex.} %definizione ambiente esempio/esercizio

\theoremsymbol{}

\theorembodyfont{\slshape}
\newtheorem{teo}{Theorem}[section] %definizione ambiente teorema
\newtheorem{teodef}[teo]{Thm./Def.} %definizione ambiente teorema
\newtheorem{prop}[teo]{Proposition}    %definizione ambiente proposizione
\newtheorem{cor}[teo]{Corollary}       %definizione ambiente corollario
\newtheorem{defin}[teo]{Definition}%definizione ambiente definizione 
\newtheorem{lem}[teo]{Lemma}           %definizione ambiente lemma 
\newtheorem{prob}[teo]{Problem}           %definizione ambiente lemma 

%\newenvironment{prf}{\begin{proof}{\textsl{Dimostrazione}}}{\end{proof}} % dimostrazioni con stile cambiato

% svolgimento
\theoremstyle{nonumberplain}
\theoremheaderfont{\slshape}
\theorembodyfont{\normalfont\small}
\theoremseparator{.\,}
\theoremsymbol{$\circledS$}
\newtheorem{svolg}{Svolgimento}
%

% dimostrazione
\theoremstyle{nonumberplain}
\theoremheaderfont{\slshape}
\theorembodyfont{\normalfont\small}
\theoremseparator{.\,}
\theoremsymbol{$\blacksquare$}
\newtheorem{prf}{Proof}
%


%Osservazione
\theoremstyle{plain}
\theoremheaderfont{\bfseries}
\theorembodyfont{\normalfont}
%\theoremseparator{.\,}
\theoremsymbol{}
\newtheorem{remark}{Remark}[section]
%
%%%%%%%%



%comando teoremi importanti impTeo


%% the following is commaon for all examples in mdframed manual
%\mdfsetup{skipabove=\topskip,skipbelow=\topskip}
%%% up to here
\colorlet{ColimpTeo}{myroyal5}


%\newenvironment{impTeo}[1][]{%
%\refstepcounter{teo}%
%\ifstrempty{#1}%
%{\mdfsetup{%
%frametitle={%
%\tikz[baseline=(current bounding box.east),outer sep=0pt]
%\node[anchor=east,rectangle,draw=ColimpTeo, fill=ColimpTeo!80, linewidth=0.5mm]{\strut Teorema~\theteo};}}
%}%
%{\mdfsetup{%
%frametitle={%
%\tikz[baseline=(current bounding box.east),outer sep=0pt]
%\node[anchor=east,rectangle,draw=ColimpTeo, fill=white, line width=1mm]
%{\strut Teorema~\theteo:~#1};}}%
%}%
%\mdfsetup{innertopmargin=10pt,linecolor=ColimpTeo,%
%linewidth=2pt,topline=true,%
%frametitleaboveskip=\dimexpr-\ht\strutbox\relax, roundcorner=10pt
%}
%\begin{mdframed}[]\relax%
%}{\end{mdframed}}



\tcbuselibrary{theorems}
\newtcbtheorem[number within=section]{newese}{Esempio}%
{colback=green!5,colframe=green!35!black,,fonttitle=\bfseries}{nese}

\newtcbtheorem[use counter*=teo, number within=section]% init options
  {impTeo}% name
  {Theorem}% title
  {%
    colback=blue!5,
    colframe=ColimpTeo,
    fonttitle={\usefont{T1}{lmss}{bx}{n}},%\bfseries,
  }% options
  {impTeo}% prefix

\newtcbtheorem[use counter*=teo, number within=section]% init options
  {subimpTeo}% name
  {Theorem}% title
  {%
    colback=blue!5,
    colframe=ColimpTeo,
    fonttitle={\usefont{T1}{lmss}{bx}{n}},%\bfseries,
  }% options
  {subimpTeo}% prefix



%\newenvironment{impTeo}[1][]{%
%\refstepcounter{teo}%
%%\setcounter{impTeo}{teo}%
%\ifstrempty{#1}%
% {\mdfsetup{%
%   frametitle={%
%    \tikz[baseline=(current bounding box.east),outer sep=0pt]
%    \node[anchor=east,rectangle,fill=blue!40]
%         {\strut Teorema~\theteo};}}
% }%
%{\mdfsetup{%
%  frametitle={%
%   \tikz[baseline=(current bounding box.east),outer sep=0pt]
%   \node[anchor=east,rectangle,fill=blue!40]
%        {\strut Teorema~\theteo:~#1};}}%
% }%
%\mdfsetup{innertopmargin=10pt,
%linecolor=blue!40,%
%       linewidth=2pt
%       ,topline=true
%       ,frametitleaboveskip=\dimexpr-\ht\strutbox\relax
%       }
%   \begin{mdframed}[]\relax%
%}
%{\end{mdframed}}




%% important Lemma 

\colorlet{ColimpLem}{bord1}

%\mdfsetup{skipabove=\topskip,skipbelow=\topskip}
%%%% upto here
%\newenvironment{impLem}[1][]{%
%\refstepcounter{teo}%
%\ifstrempty{#1}%
% {\mdfsetup{%
%   frametitle={%
%    \tikz[baseline=(current bounding box.east),outer sep=0pt]
%    \node[anchor=east,rectangle,draw=ColimpLem, fill=black, linewidth=0.5mm]
%         {\strut Lemma~\theteo};}}
% }%
%{\mdfsetup{%
%  frametitle={%
%   \tikz[baseline=(current bounding box.east),outer sep=0pt]
%   \node[anchor=east,rectangle,draw=ColimpLem, fill=white, line width=1mm]
%        {\strut Lemma~\theteo:~#1};}}%
% }%
%\mdfsetup{innertopmargin=10pt,linecolor=ColimpLem,%
%       linewidth=2pt,topline=true,
%       frametitleaboveskip=\dimexpr-\ht\strutbox\relax, roundcorner=10pt}
%   \begin{mdframed}[]\relax%
%}
%{\end{mdframed}}

\newtcbtheorem[use counter*=teo, number within=section]% init options
  {impLem}% name
  {Theorem}% title
  {%
    colback=red!5,
    colframe=ColimpLem,
    fonttitle={\usefont{T1}{lmss}{bx}{n}},%\bfseries,
  }% options
  {impTeo}% prefix

\newtcbtheorem[use counter*=teo, number within=section]% init options
  {subimpLem}% name
  {Theorem}% title
  {%
    colback=red!5,
    colframe=ColimpLem,
    fonttitle={\usefont{T1}{lmss}{bx}{n}},%\bfseries,
  }% options
  {subimpLem}% prefix



\colorlet{ColimpConj}{Dark2}

\newtcbtheorem[use counter*=teo, number within=section]% init options
  {impConj}% name
  {Conjecture}% title
  {%
    colback=red!5,
    colframe=ColimpConj,
    fonttitle={\usefont{T1}{lmss}{bx}{n}},%\bfseries,
  }% options
  {impConj}% prefix



%
%\newtheoremstyle{break}
%  {\topsep}
%  {\topsep}
%  {\itshape}
%  {0pt}
%  {\bfseries}
%  {.}
%  {\newline}
%  {\thmname{#1}\thmnumber{ #2}\thmnote{ \textbf{(#3)}}}
%\theoremstyle{break}


%%%%%%%%%%%%%%%%%%%%%%%%%%%%%%%%%%%%%%%%%%
%%% - Theorems with bar


\colorlet{ColTheos}{dark1}
\colorlet{ColExamp}{dark_green1}
\colorlet{ColConj}{Dark2}


\theorembodyfont{\slshape}
\theoremheaderfont{\bfseries}
\theorempreskipamount\topsep
\theorempostskipamount\topsep
\theoremsymbol{}

\newmdtheoremenv[
  linecolor=ColTheos,
  linewidth=2pt,
  topline=false,rightline=false,bottomline=false,
  innertopmargin=0pt,
  innerbottommargin=10pt,
  innerrightmargin=0pt,
]{nteo}[teo]{Theorem}  

\newmdtheoremenv[
  linecolor=ColTheos,
  linewidth=2pt,
  topline=false,rightline=false,bottomline=false,
  innertopmargin=0pt,
  innerbottommargin=10pt,
  innerrightmargin=0pt,
]{nlem}[teo]{Lemma}  


\newmdtheoremenv[
  linecolor=ColConj,
  linewidth=2pt,
  topline=false,rightline=false,bottomline=false,
  innertopmargin=0pt,
  innerbottommargin=10pt,
  innerrightmargin=0pt,
]{nconj}[teo]{Conjecture}  
  

\newmdtheoremenv[
  linecolor=ColTheos,
  linewidth=2pt,
  topline=false,rightline=false,bottomline=false,
  innertopmargin=0pt,
  innerbottommargin=10pt,
  innerrightmargin=0pt,
]{nprop}[teo]{Proposition}  
  
  
  \newmdtheoremenv[
  linecolor=ColTheos,
  linewidth=2pt,
  topline=false,rightline=false,bottomline=false,
  innertopmargin=0pt,
  innerbottommargin=10pt,
  innerrightmargin=0pt,
]{ncor}[teo]{Corollary}  


\newmdtheoremenv[
  linecolor=ColTheos,
  linewidth=2pt,
  topline=false,rightline=false,bottomline=false,
  innertopmargin=0pt,
  innerbottommargin=10pt,
  innerrightmargin=0pt,
]{subnewprop}[teo]{Proposition}




\def\exampletext{Example} % If English
\newcounter{testexample}
\NewDocumentEnvironment{nese}{ O{} }
{
\newtcolorbox[use counter=testexample]{testexamplebox}{%
    % Example Frame Start
    empty,% Empty previously set parameters
    title={ %background = colback=blue!30
    \begin{tcolorbox}[width=13cm, colframe=ColExamp, arc=2mm, sharp corners=west, boxsep=1mm]
    \exampletext\ \thesection.\thetcbcounter: #1
    \end{tcolorbox}},% use \thetcbcounter to access the testexample counter text
    % Attaching a box requires an overlay
    attach boxed title to top left,
       % Ensures proper line breaking in longer titles
       minipage boxed title,
    % (boxed title style requires an overlay)
    boxed title style={empty,size=minimal,toprule=0pt,top=4pt,left=3mm,overlay={}},
    coltitle=ColExamp,fonttitle=\bfseries,
    before=\par\medskip\noindent,parbox=false,boxsep=0pt,left=3mm,right=0mm,top=0pt,breakable,pad at break=0mm,
       before upper=\csname @totalleftmargin\endcsname0pt, % Use instead of parbox=true. This ensures parskip is inherited by box.
    % Handles box when it exists on one page only
    overlay unbroken={\draw[ColExamp,line width=2pt] ([xshift=-0pt]title.north west) -- ([xshift=-0pt]frame.south west); },
    % Handles multipage box: first page
    overlay first={\draw[ColExamp,line width=2pt] ([xshift=-0pt]title.north west) -- ([xshift=-0pt]frame.south west); },
    % Handles multipage box: middle page
    overlay middle={\draw[ColExamp,line width=2pt] ([xshift=-0pt]frame.north west) -- ([xshift=-0pt]frame.south west); },
    % Handles multipage box: last page
    overlay last={\draw[ColExamp,line width=2pt] ([xshift=-0pt]frame.north west) -- ([xshift=-0pt]frame.south west); },%
    }
\begin{testexamplebox}}
{\end{testexamplebox}\endlist}












